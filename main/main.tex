\documentclass{article}

\usepackage{graphicx}
\usepackage{xcolor}
\usepackage{fancyhdr}
\usepackage{titlesec}
\usepackage{tcolorbox}
\usepackage{listings}
\usepackage{enumitem}
\usepackage[hidelinks]{hyperref} 
\usepackage{xepersian}

\settextfont[BoldFont={XB Zar bold.ttf}]{XB Zar.ttf}

\definecolor{darkelectricblue}{rgb}{0.33, 0.41, 0.47}
\definecolor{junglegreen}{rgb}{0.16, 0.67, 0.53}
\definecolor{celestialblue}{rgb}{0.29, 0.59, 0.82}
\definecolor{caribbeangreen}{rgb}{0.0, 0.8, 0.6}

\color{darkelectricblue}

\definecolor{brightlavender}{rgb}{0.75, 0.58, 0.89}
\definecolor{cadetgrey}{rgb}{0.57, 0.64, 0.69}
\definecolor{byzantine}{rgb}{0.74, 0.2, 0.64}
\definecolor{darkpowderblue}{rgb}{0.0, 0.2, 0.6}

\lstdefinestyle{cstyle}{
    commentstyle=\color{darkpowderblue},
    keywordstyle=\color{byzantine},
    numberstyle=\tiny\color{cadetgrey},
    stringstyle=\color{brightlavender},
    basicstyle=\ttfamily\footnotesize,
    breakatwhitespace=false,         
    breaklines=true,                 
    captionpos=b,                    
    keepspaces=true,                 
    numbers=left,                    
    numbersep=5pt,                  
    showspaces=false,                
    showstringspaces=false,
    showtabs=false,                  
    tabsize=2
}

\lstset{style=cstyle}

\renewcommand{\headrule}{\hbox to\headwidth{\color{celestialblue}\leaders\hrule height \headrulewidth\hfill}}

\pagestyle{fancy}
\rhead{\color{celestialblue}دانشکده مهندسی کامپیوتر دانشگاه صنعتی شریف}
\lhead{\color{celestialblue}فاز اول پروژه درس مبانی برنامه‌سازی}

\titleformat{\section}[display]{\bfseries\Large\itshape\color{junglegreen}}{}{0.5ex}
    {
        \rule{\textwidth}{1pt}
        \vspace{1ex}
        \centering
        \thesection.
    }
    [
        \vspace{-2ex}
        \rule{\textwidth}{0.3pt}
    ]

\titleformat{\subsection}{\normalfont\bfseries\color{caribbeangreen}}{\thesubsection}{0.5em}{}
\titleformat{\subsubsection}{\normalfont\bfseries\color{celestialblue}}{\thesubsubsection}{0.5em}{}

\newcommand{\codeline}[3]{
    \begin{tcolorbox}[colback=#3!5!white,colframe=#3!75!black,title=\textbf{#1}]
        \begin{latin}
            \lstinline[language=c]{#2}
        \end{latin}
    \end{tcolorbox}
}

\newcommand{\codeinput}[1]{
    \codeline{ورودی}{#1}{celestialblue}
}

\newcommand{\codeoutput}[1]{
    \codeline{خروجی}{#1}{caribbeangreen}
}

\begin{document}

    \begin{titlepage}
        \begin{center}
            \includegraphics{./Resources/sharif-large.png} \\
            \vspace{0.5cm}
            \Huge\textbf{فاز اول پروژه درس مبانی برنامه‌سازی} \\
            \vspace{1cm}
            \Large\textbf{دانشکده مهندسی کامپیوتر دانشگاه صنعتی شریف} \\
            \vspace{1cm}
            \Large مسئول پروژه: \textbf{کسری امانی} \\
            \vspace{1cm}
            \Large طراحان فاز اول: \textbf{کسری امانی، آرین یزدان‌پرست} \\
            \vspace{1cm}
            \Large طراحان داک: \textbf{پارسا محمدیان، کسری امانی} \\
            \vspace{1cm}
            \large نیم‌سال دوم 00-99\\
        \end{center}
    \end{titlepage}

    \thispagestyle{empty}
    \tableofcontents
    \newpage
    \clearpage
    \setcounter{page}{1}

    \section{مقدمه}
    در دنیای امروز و خصوصا به خاطر 
    پاندمی جهانی، آنلاین شاپ‌ها بسیار مهم هستند. هدف ما نیز 
    در پروژه مبانی برنامه‌نویسی پیاده‌سازی پلتفرمی نسبتا 
    ساده برای یک شاپ است تا گامی هدفمند و پایانی خوش 
    برای این درس رقم بخورد.

    برای پیاده سازی این پروژه شما باید اطلاعاتی که کالربران 
    می تانند به آن ها دسترسی داشته باشند را در قالب فایل
    ذخیره کنید تا در صورت اجرای مجدد برنامه بتوان از آن ها 
    استفاده کرد. همچنین برای ذخیره سازی اطلاعات در فایل
    بایستی از 
    \lr{JSON}
    استفاده کنید. در رابطه با 
    \lr{JSON}
    جلوتر صحبت خواهیم کرد.
    در طول پیاده سازی همواره این مطلب را در نظر داشته باشید که این پروژه 
    شما است و دستتان در شخصی سازی آن کاملا باز است، اما دقت کنید که قابلیت ها و 
    ویژگی های ذکر شده در داک را پیاده کنید. چگونگی پیاده سازی آن 
    به عهده خودتان است.

    \section{منوها}
    \subsection{منوی اصلی}
    در ابتدای برنامه کاربر با منویی مواجه می شود که 
    سه گزینه دارد:
    \begin{itemize}[label=\textcolor{celestialblue}{\textbullet}]
        \item \lr{Sign Up}
        \item \lr{Login}
        \item \lr{Exit}
    \end{itemize}
    \subsubsection{\lr{Sign Up}}
    در صورت انتخاب گزینه 
    \lr{Sign Up}
    در صفحه ی بعد کاربر بایستی اطلاعات خود را که شامل 
    \begin{itemize}[label=\textcolor{celestialblue}{\textbullet}]
        \item \lr{username}
        \item \lr{password}
        \item \lr{first name}
        \item \lr{last name}
        \item \lr{role}
        \item \lr{starting money}
    \end{itemize}
    است را وارد کند.
    اگر 
    \lr{username}
    تکراری است، برنامه باید خطای مناسب را برگرداند و 
    در صفحه ثبت نام بماند.
    اگر ثبت نام موفقیت آمیز بود، پیغام مناسب نمایش داده
    شود و کاربر به منوی اصلی برگردانده شود.
    قسمت
    \lr{role} 
    در بخش ثبت نام سه حالت دارد:
    \begin{itemize}[label=\textcolor{celestialblue}{\textbullet}]
        \item \lr{Customer}
        \item \lr{Retailer}
        \item \lr{Admin}
    \end{itemize}
    که هرکدام وظایف و قابلیت های متفاوتی دارند.

    \subsubsection{\lr{Login}}
    گزینه 
    \lr{Login}
    در منوی اصلی کاربر را به صفحه مربوطه می برد و در آن صفحه،
    یوزرنیم و پسورد از کاربر خواسته می شود.
    در صورت غلط بودن هریک از این دو، پیغام مناسب نمایش 
    داده شود و کاربر به منوی اصلی برگردد.
    در صورت صحیح بودن اطلاعات ورودی، کاربر به صفحه بعدی 
    هدایت شود که بسته به 
    \lr{Role} 
    کاربر، آن صفحه متفاوت خواهد بود. در ادامه صفحه کاربری شرح داده شده است.

    \subsubsection{\lr{Exit}}
    گزینه 
    \lr{Exit}
    در منوی اصلی، به برنامه خاتمه می دهد.
    
    \subsection{صفحه‌ی کاربری}
    همانطور که گفته شد کاربر پس از 
    \lr{Login} 
    با توجه به 
    \lr{Role} 
    وارد صفحه کاربری می‌شود.
    \subsubsection{\lr{Customer}}
    هر مشتری در صفحه کاربری خود می تواند از میان گزینه های 
    زیر انتخاب کند:
    \vspace{0.5cm}
    \\ \textbf{مشاهده مشخصات:} \\
    در صورت انتخاب این گزینه کاربر وارد صفحه ای می شود 
    که در آن نام و نام خوانوادگی، تعداد خرید های موفق 
    و موجودی حسابش قابل مشاهده است.
    بدیهی است که کاربر باید بتواند به منوی قبل برگردد.
    \vspace{0.5cm}
    \\ \textbf{مشاهده محصولات:} \\
    در این صفحه باید به کاربر لیستی از محصولات موجود نمایش
    داده شود. هر محصول باید شامل اطلاعات زیر باشد:
    \begin{itemize}[label=\textcolor{celestialblue}{\textbullet}]
        \item نام محصول
        \item قیمت
        \item کتگوری (در چه دسته‌بندی قرار می‌گیرد)
        \item نام فروشنده محصول
        \item تعداد قابل عرضه
    \end{itemize}
    همچنین باید دو گزینه بازگشت و افزودن به سبد در اختیار
    کاربر باشد.
    با انتخاب گزینه افزودن به سبد، باید از کاربر تعداد محصولی که می خواهد 
    به سبد بیفزاید پرسیده شود و در صورتی که آن تعداد موجود
    نبود پیغام مناسب نمایش داده شود. در غیر این صورت 
    پیغام مناسب نمایش داده شده و کاربر به صفحه محصولات بازگردد.
    \vspace{0.5cm}
    \\ \textbf{تاریخچه خریدهای موفق کاربر:} \\
    کاربر باید به صفحه ای شامل لیستی از خریدهای گذشته 
    خود هدایت شود که شامل لیست کالاهای خریداری شده، تعداد
    خریده شده از هر کالا و مبلغ پرداختی می باشد.
    \vspace{0.5cm}
    \\ \textbf{خروج کاربر:} \\    
    کاربر از حساب خود 
    \lr{Log out}
    شود و به منوی اصلی برگردد.

    \subsubsection{\lr{Retailer}}
    نقش کاربر، فروشنده کالا است. کاربر باید در صفحه کاربری 
    خود گزینه های زیر را داشته باشد:
    \vspace{0.5cm}
    \\ \textbf{مشاهده مشخصات کاربری:} \\    
        عملکرد آن مشابه همین گرینه برای مشتری ها می باشد.
    \vspace{0.5cm}
    \\ \textbf{مشاهده کتگوری های موجود:} \\
    لیستی از کتگوری برای کالاها نشان داده می شود. بدیهی
    است که باید گزینه ای برای بازگشت به منوی قبلی وجود داشته 
    باشد.   
    \vspace{0.5cm}
    \\ \textbf{ایجاد کالای جدید:} \\
    در این قسمت فروشنده باید بتواند با تعیین مشخصات یک کالا آن را به
    لیست کالاهای قابل خریداری توسط یک مشتری بیفزاید. این
    مشخصات شامل موارد زیر می شوند:
    \begin{itemize}[label=\textcolor{celestialblue}{\textbullet}]
        \item نام کالا
        \item قیمت کالا
        \item کتگوری که به آن تعلق دارد
        \item تعداد موجود از آن کالا
    \end{itemize}
    در صورت وجود نداشتن کتگوری، پیغام مناسب برگردانده شود.
    در غیر این صورت پیغام موفقیت عملیات به کاربر نمایش 
    داده شود و کاربر به صفحه پنل کاربری بازگردانده شود.
    \vspace{0.5cm}
    \\ \textbf{لیست فروش:} \\
    باید شامل نام خریدار، نام محصول خریداری شده، تعداد 
    خریداری شده و مبلغ پرداخت شده باشد.
    بدیهی است که باید  راهی برای بازگشت کاربر به منوی 
    قبل قرار دهید.
    \subsubsection{\lr{Admin}}
    وظیفه ادمین سیستم، نظارت و مدیریت آن است. پس از لاگین، کاربر 
    ادمین با گزینه های زیر مواجه می شود:
    \vspace{0.5cm}
    \\ \textbf{مشاهده مشخصات:} \\    
    مشابه همین گزینه در منوی مشتری عمل می کند با این تفاوت که 
    میزان موجودی را نمی توان مشاهده کرد(برای ادمین سیستم موجودی معنایی ندارد)
    بدیهی است که باید راهی برای بازگشت کاربر به منوی قبلی طراحی شود.
    \vspace{0.5cm}
    \\ \textbf{مشاهده لیست کاربران:} \\    
    لیستی از کاربران به همراه نقش آنان نمایش داده می شود.
    بدیهی است که باید راهی برای بازگشت کاربر به منوی قبلی طراحی شود.
    \vspace{0.5cm}
    \\ \textbf{بن کردن کاربران:} \\    
    هر ادمین می تواند کاربران دیگر (به جز بقیه ادمین ها) را بن 
    کند. در صورتی که کاربر بن شده اقدام به لاگین کند، با پیغامی 
    مواجه می شود که وی را از این اتفاق مطلع سازد و سپس می تواند 
    به منوی اصلی برنامه بازگردد.
    بعد از انتخاب گزینه بن، کاربر به صفحه ای منتقل می شود
    که در آن باید نام کاربری فردی که می خواهد بن شود را مشخص کند.
    اگر نام کاربری وجود نداشت و یا متعلق به ادمین دیگری بود، ارور مناسب
    برگردانده شود.
    \vspace{0.5cm}
    \\ \textbf{آن‌بن کردن کاربران:} \\
    همانند بخش قبل، هر ادمین می تواند کاربر بن شده را از این وضعیت 
    خارج کند. برای این مار، کاربر را به صفحه ای هدایت کنید که در آن 
    نام کاربری کاربر مورد نظر دریافت شود. سپس اگر آن کاربر قبلا
    بن نشده و یا وجود ندارد، ارور مناسب را برگردانده و در غیر 
    این صورت وی را از حالت بن شده خارج نمایید.
    
    \vspace{1cm}

    \begin{tcolorbox}[colback=caribbeangreen!5!white,colframe=caribbeangreen!75!black,title=\textbf{نکته یک}]
        برای همه کاربران باید گزینه خروج از حساب کاربری را پیاده کنید.
    \end{tcolorbox}

    \begin{tcolorbox}[colback=caribbeangreen!5!white,colframe=caribbeangreen!75!black,title=\textbf{نکته دو}]
        برنامه را طوری طراحی کنید که هیچگاه یوزر در یکی از منوها گیر 
        نکند و بتواند به عقب باز گردد.
    \end{tcolorbox}


    \section{\lr{cJSON}}
        
    \lr{JSON (Javascript Object Notation)}
    یک فرمت خاص رشته است و تفاوتی با بقیه 
    \lr{string} 
    ها ندارد. شما با استفاده از این فرمت خاص اطلاعات را دریافت و 
    ذخیره می کنید.
    برای استفاده از 
    \lr{JSON}
    می توانید 
    \lr{cJSON} 
    را از 
    \href{https://github.com/DaveGamble/cJSON}{این} \footnote{\lr{https://github.com/DaveGamble/cJSON}}
     و 
    \href{https://sourceforge.net/projects/cjson/}{این} \footnote{\lr{https://sourceforge.net/projects/cjson/}}
    لینک
    دریافت کرده و آن را در پروژه خود به کار ببرید.
    دقت کنید که برای استفاده از آن باید دو فایل 
    \lr{cSJON.c}
    و
    \lr{cJSON.h}
    را در فولدر پروژه تان قرار دهید و هدر آن را 
    \lr{include} 
    کنید:
    \codeline{افزودن \lr{cJSON} به پروژه}{\#include <cJSON.h>}{green}
    
    در فایل 
    \lr{README} 
    لینک اول اطلاعاتی در مورد 
    \lr{cJSON}
    قرار گرفته است. برای توضیحات بیشتر از 
    \href{https://www.google.com/}{گوگل}
    استفاده کنید!
\end{document}