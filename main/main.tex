\documentclass{article}

\usepackage{graphicx}
\usepackage{xcolor}
\usepackage{fancyhdr}
\usepackage{titlesec}
\usepackage{tcolorbox}
\usepackage{listings}
\usepackage{xepersian}

\settextfont[BoldFont={XB Zar bold.ttf}]{XB Zar.ttf}

\definecolor{darkelectricblue}{rgb}{0.33, 0.41, 0.47}
\definecolor{junglegreen}{rgb}{0.16, 0.67, 0.53}
\definecolor{celestialblue}{rgb}{0.29, 0.59, 0.82}
\definecolor{caribbeangreen}{rgb}{0.0, 0.8, 0.6}

\color{darkelectricblue}

\definecolor{brightlavender}{rgb}{0.75, 0.58, 0.89}
\definecolor{cadetgrey}{rgb}{0.57, 0.64, 0.69}
\definecolor{byzantine}{rgb}{0.74, 0.2, 0.64}
\definecolor{darkpowderblue}{rgb}{0.0, 0.2, 0.6}

\lstdefinestyle{cstyle}{
    commentstyle=\color{darkpowderblue},
    keywordstyle=\color{byzantine},
    numberstyle=\tiny\color{cadetgrey},
    stringstyle=\color{brightlavender},
    basicstyle=\ttfamily\footnotesize,
    breakatwhitespace=false,         
    breaklines=true,                 
    captionpos=b,                    
    keepspaces=true,                 
    numbers=left,                    
    numbersep=5pt,                  
    showspaces=false,                
    showstringspaces=false,
    showtabs=false,                  
    tabsize=2
}

\lstset{style=cstyle}

\renewcommand{\headrule}{\hbox to\headwidth{\color{celestialblue}\leaders\hrule height \headrulewidth\hfill}}

\pagestyle{fancy}
\rhead{\color{celestialblue}دانشکده مهندسی کامپیوتر دانشگاه صنعتی شریف}
\lhead{\color{celestialblue}پروژه درس مبانی برنامه‌سازی}

\titleformat{\section}[display]{\bfseries\Large\itshape\color{junglegreen}}{}{0.5ex}
    {
        \rule{\textwidth}{1pt}
        \vspace{1ex}
        \centering
    }
    [
    \vspace{-2ex}
    \rule{\textwidth}{0.3pt}
    ]

\titleformat{\subsection}[display]{\normalfont\bfseries\color{caribbeangreen}}{}{0.5em}{}

\newcommand{\codeline}[3]{
    \begin{tcolorbox}[colback=#3!5!white,colframe=#3!75!black,title=\textbf{#1}]
        \begin{latin}
            \lstinline[language=c]{#2}
        \end{latin}
    \end{tcolorbox}
}

\newcommand{\codeinput}[1]{
    \codeline{ورودی}{#1}{celestialblue}
}

\newcommand{\codeoutput}[1]{
    \codeline{خروجی}{#1}{caribbeangreen}
}

\begin{document}

    \begin{titlepage}
        \begin{center}
            \includegraphics{./Resources/sharif-large.png} \\
            \vspace{0.5cm}
            \Huge\textbf{پروژه درس مبانی برنامه‌سازی} \\
            \vspace{1cm}
            \Large\textbf{دانشکده مهندسی کامپیوتر دانشگاه صنعتی شریف} \\
            \vspace{1cm}
            \large نیم‌سال دوم 00-99\\
            \vspace{1cm}
            \Large استاد: \textbf{فکوری} \\
            \vspace{1cm}
            \Large محلت ارسال: \textbf{30 خرداد}
        \end{center}
    \end{titlepage}

    \thispagestyle{empty}
    \tableofcontents
    \newpage
    \clearpage
    \setcounter{page}{1}

    \section{مقدمه}
    هی

    \section{جزئیات}
    هو
    \subsection{پیاده‌سازی}
    سلام

    \codeinput{create new account parsa 1234}
    \codeline{یه تیکه کد}{int a = 4; while(a-- > 0) printf("\%d", a);//print loop}{red}
    \codeoutput{account created successfully}

\end{document}